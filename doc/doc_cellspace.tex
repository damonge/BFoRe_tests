\documentclass[a4paper,10pt]{article}
\usepackage[utf8]{inputenc}
\usepackage{amsmath}
\usepackage{amssymb}
\usepackage{fullpage}
\usepackage{graphicx}
\newcommand{\nv}{\hat{\bf n}}
\newcommand{\tv}{\vec{\theta}}
\newcommand{\dpar}{\delta_\parallel}

%opening
\title{Foreground cleaning in power-spectrum space}
\author{John Locke}

\begin{document}

\maketitle

\section{Foreground model}
Let ${\bf a}_{\ell m}^i$ be the SHT of the $i$-th channel map with frequency $\nu_i$. We represent it as a vector with
components $T$, $E$ and $B$, and we model it as:
\begin{equation}
 {\bf a}_{\ell m}^i=\sum_{\alpha}{\bf b}_{\ell m}^\alpha\,f_\alpha^i,
\end{equation}
where $\alpha$ runs over three different components: CMB ($\alpha=c$), synchrotron ($\alpha=s$)
and dust ($\alpha=d$). $f^i_\alpha$ is the frequency evolution of the $\alpha$-th component
in the $i$-th frequency channel, which we model as:
\begin{equation}
  f_c(\nu)=\frac{x^2\,e^x}{(e^x-1)^2},\hspace{12pt}
  f_s(\nu)=\left(\frac{\nu}{\nu_*^s}\right)^{\beta_s},\hspace{12pt}
  f_d(\nu)=\left(\frac{\nu}{\nu_*^d}\right)^{\beta_d+1}
  \frac{\exp[h\nu_*^d/(k_BT_d)]-1}{\exp[h\nu/(k_BT_d)]-1},
\end{equation}
where $x\equiv h\nu/(k_B\,T_{\rm CMB})$.

The cross-power spectrum of all channels is then:
\begin{equation}
 \langle {\bf a}^i_{\ell m}({\bf a}^j_{\ell m})^\dag\rangle\equiv \mathsf{C}^{ij}_\ell=
 \sum_{\alpha_1,\alpha_2}\mathsf{C}^{\alpha_1,\alpha_2}_\ell\,
 f^{i}_{\alpha_1}f^{j}_{\alpha_2},
\end{equation}
where we model the cross-power spectrum of the different components as:
\begin{align}
 & \mathsf{C}_\ell^{c,c}=\text{bandpowers or parametrized model (CAMB)},\\
 & \mathsf{C}_\ell^{c,\alpha_2\neq c}=\mathsf{C}_\ell^{\alpha_1\neq c,c}=0,\\
 & \mathsf{C}_\ell^{\alpha_1=(s,d),\alpha_2=(s,d)}=\text{bandpowers or parametrized model (power law?)}.
\end{align}

\subsection{Degrees of freedom}
Let there be $N_\nu$ frequency channels and $N_c$ components, $N_p$ polarization channels, and assume that the power
spectra have been measured in $N_\ell$ bandpowers. The total number of degrees of freedom would be
\begin{equation}
  \#{\rm dof}=N_\ell\left[N_\nu N_p(N_\nu N_p+1)/2-(\#{\rm spec})-(\#{\rm param\,CMB})-N_c\times2\right]
\end{equation}
if one uses parametrized models for the component power spectra, and
\begin{equation}
  \#{\rm dof}=N_\ell\left[N_\nu N_p(N_\nu N_p+1)/2-N_c N_p(N_c N_p+1)/2-(\#{\rm spec})\right]
\end{equation}
if bandpowers are used instead. In the latter case, the free parameters of the model are:
\begin{equation}
  \theta\equiv\beta_s,\,\beta_d,\,T_d,\,\mathsf{C}^{cc}_{\ell_k},\,\mathsf{C}^{ss}_{\ell_k},\,\mathsf{C}^{sd}_{\ell_k},\,\mathsf{C}^{dd}_{\ell_k}.
\end{equation}


\section{Likelihood}
We will assume (wrongly) a Gaussian likelihood for the parameters:
\begin{equation}
 \chi^2\equiv-2\log p(\theta|\hat{\mathsf{C}}^{ij}_{\ell_k})=\sum_{k,k'}\sum_{a,b}
 [\mathsf{C}^{a}_{\ell_k}(\theta)-\hat{\mathsf{C}}^{a}_{\ell_k}]\,(M^{-1})^{a,b}_{k,k'}
 [\mathsf{C}^{b}_{\ell_{k'}}(\theta)-\hat{\mathsf{C}}^{b}_{\ell_{k'}}]
\end{equation}
where we have labelled each bandpower by an index $k$ and all quantities labelled by ${\ell_k}$ are averaged over the corresponding
bandpower. The indices $a$ and $b$ label here all possible combinations of frequency and polarization channels. Here $M$ is the
covariance matrix of the power spectrum, given below, and $\mathsf{C}_\ell(\theta)$ is the model power spectra
as described above.

\subsection{Covariance}
Here we will assume a theoretical Gaussian covariance matrix, given by:
\begin{equation}
 M^{(iP_i,jP_j),(pQ_p,qQ_q)}_{k,k'}\equiv\left\langle\Delta\hat{\mathsf{C}}^{iP_i,jP_j}_{\ell_k}\,\Delta\hat{\mathsf{C}}^{pQ_p,qQ_q}_{\ell_{k'}}\right\rangle=
 \frac{\left[\mathsf{C}_{\ell_k}^{iP_i,pQ_p}\mathsf{C}_{\ell_k}^{jP_j,qQ_q}+\mathsf{C}_{\ell_k}^{iP_i,qQ_q}\mathsf{C}_{\ell_k}^{jP_j,pQ_p}\right]}{(2\ell_k+1)\Delta\ell_kf_{\rm sky}}\delta_{kk'},
\end{equation}
where $\Delta{\ell_k}$ is the bandpower width, and $P_i,\,Q_i$ label the polarization channels. We have assumed that the bandpowers are wide enough so
that correlations between different bandpowers caused by the incomplete sky coverage are negligible, and that its effects can be
encapsulated in the factor $f_{\rm sky}$. We will estimate the covariance matrix using the power spectra measured from the data.

\end{document}
