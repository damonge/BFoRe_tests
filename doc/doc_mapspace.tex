\documentclass[a4paper,10pt]{article}
\usepackage[utf8]{inputenc}
\usepackage{fullpage}
\usepackage{amsmath}
\newcommand{\nv}{{\bf n}}
\newcommand{\wtj}[6]{\left(\begin{array}{ccc} #1 & #2 & #3\\#4 & #5 & #6\end{array} \right)}


%opening
\title{Bayesian foreground cleaning}
\author{Dr. Frankenstein}

\begin{document}

\maketitle

\section{Setup}
Model:
\begin{equation}
  {\bf d}=\hat{F}\cdot{\bf T}+{\bf n}
\end{equation}

\begin{itemize}
 \item ${\bf d},\,{\bf n}\,\leftarrow (N_p\,N_c\,N_\nu)$: data and noise.
 \item $\hat{N}\equiv\langle{\bf n}\cdot{\bf n}^T\rangle\,\leftarrow (N_p\,N_c\,N_\nu)\times(N_p\,N_c\,N_\nu)$: inverse noise variance.
 \item ${\bf T}\,\leftarrow (N_p\,N_c\,N_a)$: component amplitudes.
 \item $\hat{F}\,\leftarrow (N_p\,N_c\,N_\nu)\times(N_p\,N_c\,N_a)$: component frequency dependence.
\end{itemize}
Where:
\begin{itemize}
 \item $N_\nu$: number of frequency channels
 \item $N_p$: number of pixels
 \item $N_c=3$: number of polarization channels
 \item $N_a=3$: number of components (CMB, synchrotron and dust).
 \item The spectral dependence is:
 \begin{equation}
  F_{(p'\,c'\,\nu),(p\,c\,a)}\equiv f_a(\nu;{\bf b}_a^c(p))\,\delta_{p,p'}\delta_{c,c'}
 \end{equation}
 and ${\bf b}$ is a set of spectral parameters. Explicitly:
 \begin{align}
   {\rm CMB}&:\,f_C(\nu)=B_\nu(\Theta_{\rm CMB}),\\
   {\rm synch.}&:\,f_s(\nu;\beta_s)=\left(\frac{\nu}{\nu_0^s}\right)^{\beta_s}\\
   {\rm dust.}&:\,f_d(\nu;\beta_d,\theta_d)=\left(\frac{\nu}{\nu_0^d}\right)^{\beta_d}\frac{B_\nu(\theta_d)}{B_{\nu_0^d}(\theta_d)}
 \end{align}
 \item The noise variance is diagonal in frequency and polarization channels:
 \begin{equation}
   N_{(p\,c\,\nu),(p'\,c'\,\nu')}=N_{p,p'}^{c\nu}\delta_{cc'}\delta_{\nu\nu'}
 \end{equation}
\end{itemize}

This method consists on sampling the distribution of the parameters:
\begin{itemize}
 \item ${\bf T}\rightarrow 3\times3\times N_p$ amplitudes.
 \item ${\bf b}^c_a: \beta_s^c(p),\,\beta_d^c(p),\,\theta_d^c(p)$ spectral indices. If they were completely free they'd
       be $3\times3\times N_p$ free parameters, and therefore the system could easily be overparametrized. We will use the
       following simplifying assumptions:
       \begin{enumerate}
        \item Tier 1: The spectral parameters vary only over larger pixels $N_p'\ll N_p$.
        \item Tier 1: The spectral parameters are the same for $Q$ and $U$.
        \item Tier 2: The spectral parameters are the same for $T$ and $P$.
        \item Tier 3: The dust temperature is constant across the sky.
       \end{enumerate}
\end{itemize}
Thus, in total we have $9\times N_p+6\times N_p'$ parameters for Tier 1, $9\times N_p+3\times N_p'$ for Tier 2 and $9\times N_p+2\times N_p'+1$ for Tier 3.

\section{Sampling the likelihood}
The joint posterior is:
\begin{equation}
 p({\bf T},{\bf b}|{\bf d})\propto p_l({\bf d}|{\bf T},{\bf b})\,p_p({\bf T},{\bf b}),
\end{equation}
where $p_p$ is the prior and $p_l$ is the Gaussian likelihood, given by
\begin{equation}
  -2\log p_l({\bf d}|{\bf T},{\bf b})=c+\left[{\bf d}-\hat{F}({\bf b})\cdot{\bf T}\right]^T\cdot\hat{N}\cdot\left[{\bf d}-\hat{F}({\bf b})\cdot{\bf T}\right]
\end{equation}

The Gibbs-sampling algorithm is:
\begin{enumerate}
 \item ${\bf T}_{n+1}\leftarrow p({\bf T}|{\bf d},{\bf b}_n)\propto p({\bf d}|{\bf T},{\bf b}_n)\,p({\bf T})$
 \item ${\bf b}_{n+1}\leftarrow p({\bf b}|{\bf d},{\bf T}_{n+1})\propto p({\bf d}|{\bf T}_{n+1},{\bf b})\,p({\bf b})$ 
\end{enumerate}
{\bf Obs:} I'm not sure about the second equality in both cases.

\subsection{Sampling the amplitudes ${\bf T}$}
In this case the probability distribution is 
\begin{equation}
 -2\log p({\bf T}|{\bf d},{\bf b})=c+({\bf d}-\hat{F}{\bf T})^T\hat{N}^{-1}({\bf d}-\hat{F}{\bf T})+{\bf T}^T\hat{N}_P^{-1}{\bf T},
\end{equation}
where $\hat{N}_P$ is the prior matrix. The distribution is therefore Gaussian in ${\bf T}$, and hence the amplitudes can be sampled analytically from
\begin{equation}
 {\bf T}\leftarrow N\left({\bf m}=\hat{C}\hat{F}^T\hat{N}^{-1}{\bf d}\,;\,\hat{C}\right),
 \hspace{12pt}\hat{C}\equiv(\hat{F}^T\hat{N}^{-1}\hat{F}+\hat{N}_P^{-1})^{-1}
\end{equation}




\end{document}
